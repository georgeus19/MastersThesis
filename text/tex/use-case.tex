\chapter{Use cases}

\section{Find Fitting Vocabulary Resources}

\subsection{Description}
User has a structured graph data and they would like to map the data to resources from registered vocabularies. User presents a graph of entities and their relations. The user can also specify other search parameters. The system returns a best-effort mapping options of vocabulary resources and predicates to specified entities and relationships between them. The user can then select the the most fitting option or change the input specification to make the system search again. The user can select which resources or whole vocabularies should not be included in the search.

The system tracks the last inputs and outputs so user does not have to repeat the search again if they want to go back. 

\subsection{Input}
The specification of entities and relationships within the input graph can be done multiple ways:
\begin{itemize}
    \item Provide string tags describing an entity or a relation.
    \item Provide already defined resource (IRI) as either a result value which stays in the result.
    \item Provide already defined resource (IRI) as a hint from which to gather tags.
    \item Provide implicit relation or entity which can be matched by anything.
\end{itemize}

The system contains default rules for mapping entities and relations. The user can remove some rules or add multiple. Rules typically specify what general descriptive properties are used for tag matching or how different the output model can be from the input one. The rule can be specified for all entities or for any one entity. User can also which domains the vocabularies should not be about so they can be filtered out beforehand or whether to search through all vocabularies.

\subsection{Output}
The output should provide the most fitting mappings for the input graph of entities and resources.

\subsection{Scenario}
\begin{enumerate}
    \item 
\end{enumerate}


section{Map }
